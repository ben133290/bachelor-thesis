% !TEX root = ../Thesis.tex
\chapter{Background}

This section provides background explanations for key concepts as they relate to this thesis. The concepts covered are the sliding tile 
puzzle ($\ref{section:slidingtile}$) \dots

\section{Sliding Tile Puzzle} \label{section:slidingtile}

The classic sliding tile puzzle (or 15-puzzle) is a two dimensional combination puzzle composed of 15 numbered tiles and one empty tile arranged in a 
four by four square. The position of the empty tile can be swapped with adjacent tiles. The goal of the puzzle is to reach a state where the 
numbered tiles are in row-wise ascending order starting from the top left and the empty tile is at a specified position (typically the top left).

\todo{Formalize sliding tile as a planning task?}

\section{Pattern Databases}

Pattern Databases are among the most commonly used abstraction heuristics is search and planning $\cite{helmert:pdb}$. A Pattern Database is a surjective projection of multiple state variables onto so-called Patterns. Solving the planning task for these patterns provides an admissible heuristic for the original problem. 

\section{Post-Hoc Optimization}

